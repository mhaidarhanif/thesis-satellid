\section{Framework}
\label{sec:framework}

Framework is an essential or basic construction of an object, that is exist to underlie a system.
Specifically in web application development context, it is the prebuilt blueprint we can use, following most of their guidelines when building it.
This kind of framework are mostly created to be able to build an interactive web application.
An interactive web application framework allows a user to define user interface and logic of a web application and publish the web application.~\autocite{Addala:2013:InteractiveWebAppFramework}
In early development, it's really recommended to use a \ac{WAF}\index{web application framework}, a software framework to make it easier and faster to build web-enabled application.
Today, most of them are real-time feature enabled up to easy to deploy and scale to meet modern standards of web application development.

% --------------------------------------------------
\subsection{Full Stack Framework}
\label{ssec:fullstack-framework}

Full stack framework is basically a framework that covers both server and client side code and development in a single system.
In terms of effectiveness, it is more capable than a single side (just server or just client) framework since we only need one large part to organize and help our development process.
We choose full-stack frameworks because they're also have built in or bundled with various helpers such as librarires, template engines, protocol implementation, scaffolding, more integrated features, and even build system.
Most popular \ac{WAF} that is full stack with full JavaScript technologies or heavily based on Node.js are: Meteor, DerbyJS, Sails.js, MEAN.IO, and MeanJS.

% --------------------------------------------------
\subsection{Meteor}
\label{ssec:meteor}

Meteor is a full stack framework that currently and primarily using JavaScript and Node.js as its base platform.

...

%TODO Integrated environment

%TODO Reactive
%REACTIVE / DECLARATIVE PRINCIPLES

%TODO DDP
