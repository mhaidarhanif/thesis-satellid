\section{Writing Structure}
\label{sec:writing-structure}

The writing structure of this thesis is divided into five chapters:

\begin{enumerate}
\item Introduction
\item Literature Study
\item Analysis \& Design
\item Implementation \& Test
\item Closing
\end{enumerate}

\noindent In more depths, each chapter covers consecutive segments:

\begin{description}
\item [CHAPTER 1: INTRODUCTION] \hfill \\
Introducing what is going to be built based on background, assisted with research method, then scope or limit the problem that want to be solved.
\item [CHAPTER 2: LITERATURE STUDY] \hfill \\
Description and detailed information about each of the supporting knowledge materials. Those are knowledge management system, how it compare with other definition, database and system, \ac{DBMS}, \ac{NoSQL} database and its document variant. The programming language that is covered is JavaScript with its \ac{JSON} data format, Node.js platform, also a full stack framework called Meteor.
\item [CHAPTER 3: ANALYSIS \& DESIGN] \hfill \\
Analysis intent to solve problem and plan the based on logical reason and potential type of users, while design is about interaction and interface.
\item [CHAPTER 4: IMPLEMENTATION \& TEST] \hfill \\
Implementation is the documented development process and application result, while test is checking to make sure everything goes well.
\item [CHAPTER 5: CLOSING] \hfill \\
Telling conclusion and suggestion about the research up to the actual result.
\end{description}
