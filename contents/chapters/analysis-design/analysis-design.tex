\chapter{Analysis \& Design}
\label{chap:analysis-design}

\begin{figure}[ht]
  \centering
  \includegraphics[width=8cm]{\dir/include/satellid-logo.png}
  \caption{Satellid logo}
  \label{fig:satellid-logo}
\end{figure}

In this chapter, the process is first to analyze and design to implement the basic Satellid system as fast as possible.
There is a need to pass through a simple effective implementation method.
The required phases in the method are simply:

\begin{easylist}
& Analyzing solution to define basic logic and other user types
& Designing basic interaction and interface
\end{easylist}

The analysis and design were done for Satellid.
Speaking of Satellid's name, it is a transitioned word from satellite (the artifical technology or thing), an outer space object that can transmit, send, and receive information.
Those transmission, moreover, can even be knowledge that can be useful if those are facts.
Satellid is analogy of satellite for system with identified knowledge data or documents.

Satellid is like knowledge, information, or note taking apps, sites, or methods such as pen \& paper, Google Docs, Evernote, WorkFlowy, Wikipedia, CrunchBase, Silk, just plain text, and others.
It can also be said that Satellid also inspired by Google's knowledge graph and Wikipedia's profile box.
But in a slight difference perspective, Satellid can be also defined as a personal database or personal encyclopedia that only focus on the essential and structured contexts.

Previously learned and based on hypotetical theories, Satellid should be a simple and easy system to handle specifically and focus towards knowledge management use.
It should cover various context forms, by using the flexible data format \ac{JSON}.
The vision is to make its main usage as ubiquitous as it can be.
If Satellid is compared against its most similar or popular tools in terms of storing data, information, and knowledge, \autoref{table:kms-comparison} can summarize the difference.
The comparison is used to be the complying standard of the system.
Note that not every tools mentioned are genuinely made to be a knowledge manager, but users could decide them inderectly from an encyclopedia, note taking app, or data platform, to be a knowledge manager.

\begin{table}[h!]
\centering
\caption{KMS-related tools comparison}
\label{table:kms-comparison}
\begin{tabular}{ |c||c|c|c|c|c| }
\hline
Char.  & Satellid   & Evernote      & Google Keep & Wikipedia    & Silk \\ \hline
\hline
\shortstack{Usage\\Focus} & Knowledge  & Notes         & Notes       & Encyclopedia & Data \\ \hline
\shortstack{Base\\Form}   & Context    & Free-form     & \shortstack{Mostly\\free-form} & \shortstack{Free-form,\\Hierarchical,\\\& Linked} & \shortstack{Free-form\\\& Tabular} \\ \hline
\shortstack{Data\\Format} & JSON       & ENML          & JSON        & XML/Wiki     & JSON \\ \hline
\shortstack{Main\\Usage} & Ubiquitous & \shortstack{Business\\Work} & Casual Use  & \shortstack{Public,\\Academic} & Visualization \\
\hline
\end{tabular}
\end{table}

The characteristics define each tools:
\begin{inparaenum}[\itshape 1\upshape)]
\item usage focus: the primary substance usage of it,
\item base form: the base form feature of the system when handles the data from user,
\item data format: the primary system to store and exchange data inside or outside the software, and
\item main usage: both the initial and real use case on users.
\end{inparaenum}
As an important note, the initial version of Satellid may not be complete as the others currently have.
For now the system still only processes text data as the main features.
In addition to the compared features, Satellid also available as an open project.
Furthermore, high local only or offline support is the first class priority.
Therefore, a user could just own their own Satellid without having to connect with the central official Satellid.
This gives a lot of freedom in terms of data liberty and experience usage.
This could conclude that Satellid is similar to any mentioned tools or system, but very different in philosophical and development direction.
