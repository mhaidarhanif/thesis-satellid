\section{Logic}
\label{sec:logic}

The logic mainline is compiled up from the ``Key Reason'', ``Basic Logic'', and other ``User Types''.

% --------------------------------------------------
\subsection{Primary Logic}

Since the primary reason to build the implementation of Satellid is to enable knowledge management and sharing, together in a single hub without having to manually share it.
The focus it to make exchanging knowledge between colleagues easy.
The primary logical functionality is to enable a personal knowledge worker who frequently get knowledge or wanted to store their knowledge, without having to bother the properties arrangement because everything is managed in a predefined context.
The application could be just deployable local network (easily also in their own computer) or accessed from the Web via Internet.
In a very simple way of interaction, so it should be very intuitive.
\ac{CRUD} and search operation is a must, and could be done easily.
To keep the process faster and shorter, all of the functional process can be done in one single interface layout of the application.
So there is no need to switch back and forth to different views of search, read, create, edit, and delete.
It all could happen with a simplified unified interface.

\noindent There are some terminologies used in this logic, there are:

\begin{description}
\item [User]: The primary person who use Satellid
\item [System]: The Satellid application
\item [Database]: The databased that used by Satellid
\item [Knowledge]: The knowledge inside database that accessed and managed with Satellid
\item [Template]: The template to wrap the data schema
\item [Search bar]: The input area that User can search for knowledge
\item [Knowledge card]: The entity of knowledge, its object content
\item [Collection cards]: The output area that User can view the knowledge cards
\end{description}

\noindent The System will do the following logics:

\begin{enumerate}
\item System can be used by the User
\item System can run the ordered interaction functionality by the User
\item System constantly monitor modification of Knowledge in the Database
\end{enumerate}

\noindent The User interactions are defined by these logics:

\begin{easylist}[enumerate]
& User can use, turn on and turn off the System
& User can interact with the System to:
  && Search a stored Knowledge
  && Add a new Knowledge
  && Read a stored Knowledge
  && Edit a stored Knowledge
  && Delete a stored Knowledge
& When User enter a text into the search bar:
  && Search a Knowledge based on typed text string
  && System search for matching text
  && If matched Text is found, System show the search result
  && User get the found Knowledge
& When User click/tap the add button beside the search bar:
  && Add a new knowledge
  && User complete the new Knowldge
  && If new Knowledge is successfully inputted, System store the new Knowledge into the System
& When User click/tap the edit button beside the knowledge card:
  && Edit an existing knowledge
  && User edit the Knowldge
  && If the Knowledge is successfully edited, System store the edited Knowledge into the System
& When User click/tap the delete button beside the knowledge card:
  && Delete an existing knowledge
  && If the Knowledge is confirmed to be deleted, System delete the Knowledge from the System
\end{easylist}

% --------------------------------------------------
\subsection{Cases of User Types}

Using Satellid approach to do knowledge management should also consider to enable features corresponds to the expected and other cases of user types or user stories that listed here:

\begin{easylist}[itemize]
  & As a new user, I need to quickly use and understand the workflow, so I can immediately store my knowledge.
  & As an existing user, I can search for my existing stored knowledge, so I can immediately retrieve them.
  & As a person, I don't need to manually shout a new knowledge, so others can know what's the latest important knowledge I have then I can continue to do my study or work.
  & As a colleague of other user, I need to quickly know what other user have know, so I can learn from them.
  & As a student, I need to store and manage and find my knowledge easily, so I can neatly organize a preparation for study, exam, or portfolio.
  & As a teacher, I need to share my knowledge faster and effectively throughout the class or workshop, so I can focus on teaching and involving students to do the actual work precisely.
  & As a speaker, I need to liberate and outsource the presentation and discussion resource, so I with the audience can get the collaboration to live.
  & As a collaborator, I need to have the prerequisites to have, so I can work together at the same pace and level.
  & As the leader of some users in a team, I need to collect all their knowledge, so I can smarten and explain the knowledge.
  & As a recruiter of a new employee, I need to have the recruitees' knowledge in a consise and portable format, so I can compare and select who's the right candidate.
  & As a boss, I need to know what my employees are up to automatically, so I don't need to ask them rigorously.
  & As an engineer, I need to quickly construct a basic system based on my prior knowledge, so I know which are the most required and important elements.
  & As an information architect, I need to quickly architect the main partial of all information available that required, so I can structure the information within working environments.
  & As a developer, I need to be able to create a plugin or extension, so I can add more functionality to the system/platform.
  & As a designer, I need to be able to create a custom theme, so I can customize the appearance based on my style and taste.
  & As a researcher, I need to be able collect various data and research information related to my current knowledge and purpose, so I can quickly connect and filter them to be used within my experiments.
\end{easylist}

Even not mentioned explicitly again within those user types, the shape of knowledge management will normally always be the same, to make sure those knowledge are easily managed and neatly organized across different people.
