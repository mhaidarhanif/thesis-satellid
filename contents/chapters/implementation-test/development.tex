\section{Development}
\label{sec:development}

The \textit{development} process, building that involve coding and programming the software, is resulting the working application.
In addition, open sourcing the software source code is also done and described.

% --------------------------------------------------
\subsection{Code}

JavaScript and Meteor are used on both of the sides, while Node.js is used on the server side, all coded to implement the designed system.
JavaScript is chosen more solely because it is the primary language of the Web.
Meteor is chosen because it is heavily based on JavaScript and easy to learn than any other framework out there.

TODO Some snippets of server-side code

\begin{listing}[htbp]
  \caption{Satellid server code snippets}
  %TODO-CODE \inputminted{javascript}{\dir/include/satellid-server.coffee}
  \label{lst:satellid-code-server}
\end{listing}

TODO Some snippets of client-side code

\begin{listing}[htbp]
  \caption{Satellid client code snippets}
  %TODO-CODE \inputminted{javascript}{\dir/include/satellid-client.coffee}
  \label{lst:satellid-code-client}
\end{listing}

% --------------------------------------------------
\subsection{Result}

TODO Final screenshot

\begin{figure}[htb]
  \centering
  %TODO-SCREEN \includegraphics[width=\textwidth]{\dir/include/satellid-result.png}
  \caption{Final screenshot of the development result}
  \label{fig:satellid-result}
\end{figure}

% --------------------------------------------------
\subsection{Open Sourcing}

Below in listing \autoref{lst:git} is a process of creating a Git\index{Git} repository\index{repository} and releasing it as an open source\index{open source} project.
All done within terminal/\ac{CLI} and the utilization of Git \ac{SCM}.

\begin{listing}[!h]
\caption{Committing and pushing the repo with Git}
\inputminted{shell-session}{\dir/include/git.shell-session}
\label{lst:git}
\end{listing}

The repository is first initialized as a Git repository.
Then all the files are added and committed with a message.
Lastly, a remote \ac{URL} of the repository on the Web is added then pushed to there.
In this process, now the Satellid repository is available at GitHub\index{GitHub} with \ac{URL} \url{https://github.com/mhaidarh/satelid-meteor}, as a public repository or open source project.
